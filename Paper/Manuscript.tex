\documentclass[]{elsarticle} %review=doublespace preprint=single 5p=2 column
%%% Begin My package additions %%%%%%%%%%%%%%%%%%%
\usepackage[hyphens]{url}

  \journal{Building and Environment} % Sets Journal name


\usepackage{lineno} % add
\providecommand{\tightlist}{%
  \setlength{\itemsep}{0pt}\setlength{\parskip}{0pt}}

\bibliographystyle{elsarticle-harv}
\biboptions{sort&compress} % For natbib
\usepackage{graphicx}
\usepackage{booktabs} % book-quality tables
%%%%%%%%%%%%%%%% end my additions to header

\usepackage[T1]{fontenc}
\usepackage{lmodern}
\usepackage{amssymb,amsmath}
\usepackage{ifxetex,ifluatex}
\usepackage{fixltx2e} % provides \textsubscript
% use upquote if available, for straight quotes in verbatim environments
\IfFileExists{upquote.sty}{\usepackage{upquote}}{}
\ifnum 0\ifxetex 1\fi\ifluatex 1\fi=0 % if pdftex
  \usepackage[utf8]{inputenc}
\else % if luatex or xelatex
  \usepackage{fontspec}
  \ifxetex
    \usepackage{xltxtra,xunicode}
  \fi
  \defaultfontfeatures{Mapping=tex-text,Scale=MatchLowercase}
  \newcommand{\euro}{€}
\fi
% use microtype if available
\IfFileExists{microtype.sty}{\usepackage{microtype}}{}
\usepackage[left=2.5cm,right=2.5cm,top=2.5cm,bottom=2.5cm]{geometry}
\usepackage{longtable}
\usepackage{graphicx}
% We will generate all images so they have a width \maxwidth. This means
% that they will get their normal width if they fit onto the page, but
% are scaled down if they would overflow the margins.
\makeatletter
\def\maxwidth{\ifdim\Gin@nat@width>\linewidth\linewidth
\else\Gin@nat@width\fi}
\makeatother
\let\Oldincludegraphics\includegraphics
\renewcommand{\includegraphics}[1]{\Oldincludegraphics[width=\maxwidth]{#1}}
\ifxetex
  \usepackage[setpagesize=false, % page size defined by xetex
              unicode=false, % unicode breaks when used with xetex
              xetex]{hyperref}
\else
  \usepackage[unicode=true]{hyperref}
\fi
\hypersetup{breaklinks=true,
            bookmarks=true,
            pdfauthor={},
            pdftitle={An R markdown template for Elsevier journals with examples for beginners},
            colorlinks=false,
            urlcolor=blue,
            linkcolor=magenta,
            pdfborder={0 0 0}}
\urlstyle{same}  % don't use monospace font for urls

\setcounter{secnumdepth}{5}
% Pandoc toggle for numbering sections (defaults to be off)
% Pandoc header
\usepackage{setspace}
\doublespacing



\usepackage{amsthm}
\newtheorem{theorem}{Theorem}[section]
\newtheorem{lemma}{Lemma}[section]
\theoremstyle{definition}
\newtheorem{definition}{Definition}[section]
\newtheorem{corollary}{Corollary}[section]
\newtheorem{proposition}{Proposition}[section]
\theoremstyle{definition}
\newtheorem{example}{Example}[section]
\theoremstyle{definition}
\newtheorem{exercise}{Exercise}[section]
\theoremstyle{remark}
\newtheorem*{remark}{Remark}
\newtheorem*{solution}{Solution}
\begin{document}
\begin{frontmatter}

  \title{An R markdown template for Elsevier journals with examples for beginners}
    \author[CBE]{Paul Raftery\corref{c1}}
   \ead{p.raftery@berkeley.edu} 
   \cortext[c1]{Corresponding Author}
    \author[Organisation2]{Next contributor}
  
  
      \address[CBE]{Center for the Built Environment, UC Berkeley, 390 Wurster Hall,
Berkeley, CA, 94720, USA}
    \address[Organisation2]{Another organization, and their address}
  
  \begin{abstract}
  Enter the text for your abstract here
  \end{abstract}
  
 \end{frontmatter}

Keywords:\\
Key 1; Key 2; Key 3; Key 4; Key 5; Key 6 (max)

\pagebreak

\textbf{Highlights:}

\begin{itemize}
\tightlist
\item
  Add your 3 -5 highlights
\item
  As bullets here
\item
  Making sure each is less than 85 characters in length
\end{itemize}

\textbf{Graphical Abstract}\\
\includegraphics{C:/Users/praft/Documents/GitHub/rmd-example/Paper/SupplementaryMaterial/Figures/Latex_logo.png}

\pagebreak

\hypertarget{introduction}{%
\section{Introduction}\label{introduction}}

The purpose of this (very much work-in-progress) document is to provide
a complete R markdown template for an Elsevier journal submission (based
on the rticles repository {[}1{]}), along with useful examples and
packages to improve usability for folks who are just starting out with
this workflow. The eventual intent is to capture minimal examples of the
common things that authors need to do when writing papers in R markdown;
provide examples of useful packages, workflows, and tools; and provide
solutions to common issues that folks encounter.

\hypertarget{examples}{%
\section{Examples}\label{examples}}

\hypertarget{heading-level-1}{%
\section{Heading level 1}\label{heading-level-1}}

\hypertarget{heading-level-2}{%
\subsection{Heading level 2}\label{heading-level-2}}

\hypertarget{heading-level-3}{%
\subsubsection{Heading level 3}\label{heading-level-3}}

Here's how to \textbf{bold} or \emph{italic} a piece of text.

This is how you do a bullet point list:

\begin{itemize}
\tightlist
\item
  First bullet
\item
  Second bullet

  \begin{itemize}
  \tightlist
  \item
    A sub-bullet
  \item
    Another sub-bullet
  \end{itemize}
\end{itemize}

Or an ordered option:

\begin{enumerate}
\def\labelenumi{\arabic{enumi}.}
\tightlist
\item
  Item 1
\item
  Item 2

  \begin{itemize}
  \tightlist
  \item
    Item 2a
  \item
    Item 2b
  \end{itemize}
\end{enumerate}

\hypertarget{citations}{%
\subsection{Citations}\label{citations}}

Citing other literature is remarkably easy, just like this {[}2{]}. This
citation key references the tag associated with an entry in
Bibliography.bib (a BibTex file). I've found it easiest to use Zotero to
manage my library of references and to generate the BibTex file, though
any software that creates a valid BibTex file should work fine. Zotero
allows you to create a `Collection' (or folder) that gathers together
all of the references used for a particular document. When combined with
with the Better BibTex plugin, that collection can be exported to a
BibTex file that is continually updated as you add or edit references in
that Collection. Better BibTex also puts the citation key - the text
after the `@' sybmbol in the .Rmd file - on the upper right of each
entry, which is convenient for adding citations.

There's not much else involved in citing, as the references list gets
built and formatted automatically based on the selected style. The only
other issue I've had to look around to solve was figuring out how to
combine multiple citations, which is easy when you know how. {[}2,3{]}

\hypertarget{cross-referencing}{%
\subsection{Cross-referencing}\label{cross-referencing}}

This is how you refer to a figure in your text: Figure
\ref{fig:correlation}. Simply reference the title of the code chunk, and
ensure that the code chunk includes a figure caption.

\begin{figure}
\centering
\includegraphics{C:/Users/praft/Documents/GitHub/rmd-example/Paper/SupplementaryMaterial/Figures/Correlation.png}
\caption{\label{fig:correlation}Correlation. Source: XKCD, xkcd.com/552}
\end{figure}

\hypertarget{calculations-in-text}{%
\subsection{Calculations in text}\label{calculations-in-text}}

The holy grail of markdown - doing all of your calculations in the same
file so you never need to worry about updating them after
someone\footnote{Often I'm the someone, sorry CBE grad students. Also,
  look, it's an example of a footnote!} asks you to make
changes\ldots{}. again! It's as easy as pi: 3.14. Incidentally,you can
selectively override the `global' options set at the beginning, to say
for example, show more decimals: 3.1416.

This is an example of outputing the result of a calculation that you
perform within a code chunk in the document somewhere prior to the
location where you first refer to it: 36.

\hypertarget{other-packages}{%
\subsection{Other packages}\label{other-packages}}

Here are lots of packages that are useful for markdown docs. I encourage
you to search for these whenever you encounter a new thing you need to
do and to propose an addition to this repository accordingly. For
example, the Stargazer package describes linear models in a nice table.

\begin{table}[!htbp] \centering 
  \caption{Car miles per gallon} 
  \label{} 
\begin{tabular}{@{\extracolsep{1pt}}lcc} 
\\[-1.8ex]\hline 
\hline \\[-1.8ex] 
 & \multicolumn{2}{c}{\textit{Dependent variable:}} \\ 
\cline{2-3} 
\\[-1.8ex] & Highway & City \\ 
\\[-1.8ex] & (1) & (2)\\ 
\hline \\[-1.8ex] 
 Displacement & $-$2.100$^{***}$ & $-$1.300$^{***}$ \\ 
  & (0.520) & (0.340) \\ 
  & & \\ 
 Num. cylinders & $-$1.300$^{***}$ & $-$1.200$^{***}$ \\ 
  & (0.410) & (0.270) \\ 
  & & \\ 
 Year & 0.150$^{***}$ & 0.071$^{**}$ \\ 
  & (0.055) & (0.036) \\ 
  & & \\ 
 Constant & $-$259.000$^{**}$ & $-$114.000 \\ 
  & (109.000) & (72.000) \\ 
  & & \\ 
\hline \\[-1.8ex] 
Observations & 234 & 234 \\ 
R$^{2}$ & 0.620 & 0.670 \\ 
Adjusted R$^{2}$ & 0.610 & 0.670 \\ 
Residual Std. Error (df = 230) & 3.700 & 2.500 \\ 
F Statistic (df = 3; 230) & 124.000$^{***}$ & 158.000$^{***}$ \\ 
\hline 
\hline \\[-1.8ex] 
\textit{Note:}  & \multicolumn{2}{r}{$^{*}$p$<$0.1; $^{**}$p$<$0.05; $^{***}$p$<$0.01} \\ 
\end{tabular} 
\end{table}

\hypertarget{equations-and-math}{%
\subsection{Equations and math}\label{equations-and-math}}

Here's a basic example inline \(example_{subscript} = \frac{D}{R}\), or
you display it on a whole line if needed. Google latex math cheat sheets
for more information.

\[\sum_{i=1}^{n}{x_i^2}\]

\hypertarget{writing-style}{%
\section{Writing style}\label{writing-style}}

This is a little off topic for an Rmd example but a convenient place to
remind our grad students about writing style. In almost all cases,
active voice is better than passive voice. Several psychological studies
show that the active voice is more easily understood by readers, and
that information is more accurately reported by authors when writing in
active voice. For example, research {[}4{]} has shown that the ``active
{[}voice{]} offers a neutral structure for conveying information''.
Authorship guides for highly regarded journals often indicate a
preference for the active voice instead of passive:

\begin{itemize}
\tightlist
\item
  Nature: ``Nature journals like authors to write in the active voice
  (`we performed the experiment\ldots{}') as experience has shown that
  readers find concepts and results to be conveyed more clearly if
  written directly.''{[}5{]}
\item
  Science: ``Use active voice when suitable, particularly when necessary
  for correct syntax (e.g., `To address this possibility, we constructed
  a lZap library \ldots{},' not `To address this possibility, a lZap
  library was constructed\ldots{}').'' {[}6{]}
\end{itemize}

And, on top of all that, you also end up with less text if you write in
active voice, saving space for useful information and making it easier
for your readers to understand.

\hypertarget{a-fun-way-to-spot-passive-voice}{%
\subsection{A fun way to spot passive
voice:}\label{a-fun-way-to-spot-passive-voice}}

If you can add the words `by zombies' {[}7{]} to the end of the sentence
and the sentence still makes logical sense, then the sentence is in
passive voice. You can also switch on the grammar settings in Microsoft
Word's spelling and grammar checker and it will show up that way.

How to fix it?

Change:

``These measurements are not quantitatively reported in the paper''
(\ldots{} by zombies)

To

``The paper does not quantitatively report these measurements''

Or even better, it's really the authors doing the reporting as the paper
is an inanimate object\ldots{}

``We do not quantitatively report these measurements''.

Change:

``Six different table and partition configurations were tested''
(\ldots{} by zombies)

To

``We tested six different table and partition configurations.''

\hypertarget{methods}{%
\section{Methods}\label{methods}}

\hypertarget{results}{%
\section{Results}\label{results}}

\hypertarget{discussion}{%
\section{Discussion}\label{discussion}}

\hypertarget{conclusion}{%
\section{Conclusion}\label{conclusion}}

\hypertarget{acknowledgements}{%
\section{Acknowledgements}\label{acknowledgements}}

Don't forget to acknowledge the funder(s) with associated grant numbers
if required. The same goes for folks who significantly assisted you with
this paper but that are not authors.

\hypertarget{declaration-of-interest}{%
\section{Declaration of interest}\label{declaration-of-interest}}

Descibe any relevant interests of the authors, particularly if there is
a link to the research that is relatively uncommon and could be
perceived as a conflict of interest.

\hypertarget{references}{%
\section*{References}\label{references}}
\addcontentsline{toc}{section}{References}

\hypertarget{refs}{}
\leavevmode\hypertarget{ref-LaTeXJournalArticle2019}{}%
{[}1{]} LaTeX Journal Article Templates for R Markdown. Contribute to
rstudio/rticles development by creating an account on GitHub, (2019).

\leavevmode\hypertarget{ref-coakleyReviewMethodsMatch2014}{}%
{[}2{]} D. Coakley, P. Raftery, M. Keane, A review of methods to match
building energy simulation models to measured data, Renewable and
Sustainable Energy Reviews. 37 (2014) 123--141.
doi:\href{https://doi.org/10.1016/j.rser.2014.05.007}{10.1016/j.rser.2014.05.007}.

\leavevmode\hypertarget{ref-zhaiHumanComfortPerceived2015a}{}%
{[}3{]} Y. Zhai, Y. Zhang, H. Zhang, W. Pasut, E. Arens, Q. Meng, Human
comfort and perceived air quality in warm and humid environments with
ceiling fans, Building and Environment. 90 (2015) 178--185.
doi:\href{https://doi.org/10.1016/j.buildenv.2015.04.003}{10.1016/j.buildenv.2015.04.003}.

\leavevmode\hypertarget{ref-klenbortMarkednessPerspectiveInterpretation1974}{}%
{[}4{]} I. Klenbort, M. Anisfeld, Markedness and perspective in the
interpretation of the active and passive voice, Quarterly Journal of
Experimental Psychology. 26 (1974) 189--195.
doi:\href{https://doi.org/10.1080/14640747408400404}{10.1080/14640747408400404}.

\leavevmode\hypertarget{ref-NatureHowWrite}{}%
{[}5{]} Nature - How to write a paper, (n.d.).

\leavevmode\hypertarget{ref-rubenHowWriteScientist2012}{}%
{[}6{]} A. Ruben, 2012, 8. Am, How to Write Like a Scientist, Science
\textbar{} AAAS. (2012).

\leavevmode\hypertarget{ref-ScaryeasyWayHelp2014}{}%
{[}7{]} A scary-easy way to help you find passive voice!, A Scary-Easy
Way to Help You Find Passive Voice! \textbar{} Grammarly Blog. (2014).

\end{document}


